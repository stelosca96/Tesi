
\chapter{Stato dell'arte}

\section{Gli attacchi DDoS}
Prova prova

\section{I satelliti medicei}
Prova prova

\chapter{Il barometro}
\section{Generalit\`a}
\begin{interlinea}{0.87} Il barometro, come dice il nome, serve per
misurare la pesantezza; pi\`u precisamente la pesantezza dell'aria
riferita all'unit\`a di superficie.
\end{interlinea}

\begin{interlinea}{2} Studiando il fenomeno fisico si pu\`o concludere
che in un dato punto grava il peso della colonna d'aria che lo
sovrasta, e che tale colonna \`e tanto pi\`u grave quanto maggiore
\`e la superficie della sua base; il rapporto fra il peso e la base
della colonna si chiama pressione e si misura in once toscane al cubito
quadrato, \cite{tor1}; nel Ducato di Savoia la misura in once al piede
quadrato \`e quasi uguale, perch\'e col\`a usano un piede molto
grande, che \`e simile al nostro cubito.
\end{interlinea}

\subsection{Forma del barometro}
Il barometro consta di un tubo di vetro chiuso ad una estremit\`a e
ripieno di mercurio, capovolto su di un vaso anch'esso ripieno di
mercurio; mediante un'asta graduata si pu\`o misurare la distanza fra
il menisco del mercurio dentro il tubo e la superficie del mercurio
dentro il vaso; tale distanza \`e normalmente di 10 pollici toscani,
\cite{tor1,tor2}, ma la misura pu\`o variare se si usano dei pollici
diversi; \`e noto infatti che gl'huomini sogliono avere mani di
diverse grandezze, talch\'e anche li pollici non sono egualmente
lunghi.
\section{Del mercurio}
Il mercurio \`e un a sostanza che si presenta come un liquido, ma ha il colore
del metallo. Esso \`e pesantissimo, tanto che un bicchiere, che se fosse pieno
d'acqua, sarebbe assai leggiero, quando invece fosse ripieno di mercurio,
sarebbe tanto pesante che con entrambe le mani esso necessiterebbe di essere
levato in suso.

Esso mercurio non trovasi in natura nello stato nel quale \`e d'uopo che sia
per la costruzione dei barometri, almeno non trovasi cos\`i abbondante come
sarebbe necessario.

\setcounter{footnote}{25}

Il Monte Amiata, che \`e locato nel territorio del Ducato%
\footnote{Naturalmente stiamo parlando del Granducato di Toscana.%
\ifclassica\NoteWhiteLine\fi
} del nostro Eccellentissimo et Illustrissimo Signore Granduca di Toscana\footnote{Cosimo IV de' Medici.}, \`e uno dei
luoghi della terra dove pu\`o rinvenirsi in gran copia un sale rosso, che
nomasi \emph{cinabro}, dal quale con artifizi alchemici, si estrae il mercurio
nella forma e nella consistenza che occorre per la costruzione del barometro
terrestre%
\ifclassica
\nota{Nota senza numero\dots

\dots e che va a capo.
}\fi.


La densit\`a del mercurio \`e molto alta e varia con la temperatura come
pu\`o desumersi dalla tabella \ref{t:1}.


Il mercurio gode della sorprendente qualit\`a et propriet\`a, cio\`e che esso
diventa tanto solido da potersene fare una testa di martello et infiggere
chiodi aguzzi nel legname.
\begin{table}[htp]              % crea un floating body col nome Tabella nella
                            % didascalia
\centering                      % comando necessario per centrare la tabella
\begin{tabular}%                % inizio vero e proprio della tabella
{rrrrrr}                        % parametri di incolonnamento
\hline\hline                    % filetti orizzontali sopra la tabella
                            % intestazione della tabella
\multicolumn{3}{c}{\rule{0pt}{2.5ex}Temperatura} % \rule serve per lasciare
& \multicolumn{3}{c}{Densit\`a} \\               % un po' di spazio sopra le parole
&\unit{\gradi C} & & & $\unit{t/m^3}$ &  \\
\hline%                         % Filetto orizzontale per separare l'intestazione
\hspace*{1.3em}& 0  &  & & 13,8 &  \\   % I numeri sono incolonnati % 
          & 10  &  & & 13,6 &  \\   % a destra; le colonne vuote
          & 50  &  & & 13,5 &  \\   % servono per centrare le colonne
          &100  &  & & 13,3 &  \\   % numeriche sotto le intestazioni
\hline \hline                           % Filetti di fine tabella
\end{tabular}
\caption[Densit\`a del mercurio]{Densit\`a del mercurio. Si pu\`o fare molto meglio usando il pacchetto \textsf{booktabs}.} \label{t:1}  % didascalia con label
\end{table}

%\selectlanguage{italian}

\begin{osservazione}\normalfont
Questa propriet\`a si manifesta quando esso \`e estremamente freddo, come
quando lo si immerge nella salamoia di sale e ghiaccio che usano li maestri
siciliani per confetionare li sorbetti, dei quali sono insuperabili artisti.
\end{osservazione}

Per nostra fortuna, questo grande freddo, che necessita per la confetione de
li sorbetti, molto raramente, se non mai, viene a formarsi nelle terre del
Granduca Eccellentissimo, sicch\'e non vi ha tema che il barometro di mercurio
possa essere ruinato dal grande gelo e non indichi la pressione giusta, come
invece deve sempre fare uno strumento di misura, quale \`e quello che \`e
descritto cost\`i.\cite{duane1964}

\chapter*{Conlusioni}
E con questo si conclude la tesi d'esempio per una tesi magistrale con un capitolo non  numerato che si trova ancora nella main matter.

Dovrebbe essere evidente che il comando \texttt{\string\chapter*} non dovrebbe mai essere usato nella main matter, tranne eventualmente un capitoletto conclusivo e riassuntivo \emph{non strutturato}. Infatti se esso contenesse al suo interno paragrafi, sottoparagrafi e affini, questi verrebbero numerati erroneamente con il numero del capitolo precedente. 

\appendix

\chapter{Il listato del pacchetto \texttt{topcoman.sty}}
\listing{topcoman.sty}

\chapter{Seconda appendice}
Questa è la seconda appendice numerata con una lettera perché questo comando \texttt{\string\chapter} viene dopo il comando \texttt{\string\appendix}.

Le appendici vanno numerate se sono più di una e devono quindi stare nella main matter, perché nella back matter nulla viene numerato.

La bibliografia che segue non è numerata perché l'ambiente \texttt{thebibliography} compone il suo titolo con il comando \texttt{\string\chapter*} e ne manda il titolo nell'indice generale con i suoi propri comandi interni. La definizione di questo ambiente è specifica di questo macro-pacchetto \texttt{TOPtesi}.

Sarebbe meglio inserire la bibliografia dopo un comando \texttt{\string\backmatter} esplicito. La back matter è destina espressamente a una sola appendice non numerata (ci pensa da sola a non numerare le sue sezioni); alla bibliografia, a uno o più indici analitici, a glossari o nomenclature, liste di acronimi, e simili. Nulla è obbligatorio in una back matter, ma una tesi senza bibliografia non sarebbe appropriata, tanto meno una tesi magistrale priva di una bibliografia.

