% !TEX encoding = UTF-8 Unicode
% !TEX TS-program = pdflatex

%%%%%%% La riga soprastante serve per configurare gli editor
%%%%%%% TeXShop, TeXworks e TeXstudio per gestire questo file
%%%%%%% con la codifica UFF-8.
%%%%%%% Se si vuole usare un'altra codifica si veda sotto.
%%%%%%%

%%%%%%%  Esempio con molte opzioni
%%%%%%% Le opzioni nella forma "chiave=valore" sono definite
%%%%%%% perché la classe dalla versione 6.1.00 usa il pacchetto
%%%%%%% xkeyval. Vedere sulla documentazione in inglese o
%%%%%%% in italiano quali chiavi accettano valori.

%%%%%%% L'opzione per il corpo accetta qualsiasi valore, anche fratto
%%%%%%% (per esempio: corpo=11.5pt) e va sempre scritto con una
%%%%%%% unità di misura. L'utente è pregato di non esagerare con
%%%%%%% corpi normali minori di 9.5pt o maggiori di 13pt.
%%%%%%%
%%%%%%% Le opzioni per inputenc e fontenc vanno per prime.
%%%%%%% Vengono ignorate se NON si compone con pdfLaTeX. Ma
%%%%%%% questo è un esempio per pdfLaTeX.
%%%%%%%

\documentclass[%
corpo=12pt,
twoside,
stile=standard,
% oldstyle,
%    autoretitolo,
tipotesi=magistrale,
numerazioneromana,
% libro, utile per la stampa => mette il margine diverso per la rilegatura
greek,
evenboxes,
]{toptesi}
%%%%%%%%%%%%%%%%%%%%%%%%%%%%%%%%%%%%%%%%%%%%%%%%%%%%
%%%%%% Per la codifica d'entratasi può scegliere quella che si vuole,
%%%%%% ma si consiglia di preferire utf8; in ogni caso non scegliere
%%%%%% codifiche specifiche del sistema operativo.

\usepackage[utf8]{inputenc}% codifica d'entrata
\usepackage[T1]{fontenc}%    codifica dei font
\usepackage{lmodern}%        scelta dei font

% Vedere la documentazione toptesi-it.pdf per le
% attenzioni che bisogna usare al fine di ottenere un file
% veramente conforme alle norme per l'archiviabilità.


\usepackage{hyperref}

\hypersetup{%
    pdfpagemode={UseOutlines},
    bookmarksopen,
    pdfstartview={FitH},
    colorlinks,
    linkcolor={blue},
    citecolor={blue},
    urlcolor={blue}
}
%
%%%%%%% Esempio di composizione di tesi di laurea con PDFLATEX 
%
%
% Per scrivere testo fasullo in "latinorum"
\usepackage{lipsum}
%

%%%%%%% Definizioni locali
\newtheorem{osservazione}{Osservazione}% Standard LaTeX
\ExtendCaptions{italian}{Abstract}{Acknowledgements}

\begin{document}\errorcontextlines=9
%%%%%%% Questi comandi è meglio metterli dentro l'ambiente
%%%%%%% ThesisTitlePage con o senza asterisco, oppure in un file di
%%%%%%% configurazione personale. Si veda la documentazione
%%%%%%% inglese o italiana.
%%%%%%% Comunque i presenti comandi servono per comporre la
%%%%%%% tesi con i moduli di estensione standard del pacchetto
%%%%%%% TOPtesi.

\begin{ThesisTitlePage}
% Per cambiare la dicitura sopra la lista dei laureandi decommentare
% la riga seguente, cambiando le 4 parole in modo consistente
%
\TitoloListaCandidati{Studente,Studenti,Studentessa,Studentesse}
%
\ateneo{Politecnico di Torino}
%
% Non tutte le università hanno un nome proprio
% \nomeateneo{Sede di Torre sosjd}
%
% \struttura[III]{Matematica, Fisica e~Scienze Naturali}
%\Materia{Remote sensing}
\titolo{Sistemi anti-DDoS distribuiti}% per la laurea quinquennale e il dottorato
% \sottotitolo{Metodo dei satelliti medicei}% per la laurea quinquennale e il dottorato
%
%%%%%%% Corso degli studi
\corsodilaurea{Ingegneria Informatica}% per la laurea

%%%%%%% L'eventuale numero di matricola va fra parentesi quadre
%\show\Candidato
%\def\Candidato{Studente}
%\show\Candidato
\candidato{Stefano \textsc{Loscalzo}}[s267614] 
%\secondocandidato{Evangelista \textsc{Torricelli}}[123457]

%%%%%%% Relatori o supervisori
%
\relatore{prof.~Guido Marchetto}
% \secondorelatore{dipl.~ing.~Werner von Braun}
% \terzorelatore{dott.~Mario Rossi}
% 
%%%%%%% Per mettere altri relatori consultare toptesi-it.pdf

%%%%%%% Tutore
\tutoreaziendale{dott.\ ing.\ Giovanni Giacosa}
\NomeTutoreAziendale{Supervisore aziendale\\Centro Ricerche FIAT}

%%%%%%% Seduta dell'esame
%\sedutadilaurea{Agosto 1615}
%%%%%%%% oppure:
\sedutadilaurea{\textsc{Anno~accademico} 2020-2021}% 

%%%%%%% Logo della sede
% \logosede{logodue}% 
\end{ThesisTitlePage}


%%%%%%% Per cambiare l'offset per la rilegatura;
%%%%%%% meno offset c'e', meglio e'
%\setbindingcorrection{3mm}




%%%%%%% Questo test è usato appunto per collaudare diversi stili,
%%%%%%% non per comporre una vera tesi.
%%%%%%% Non usarlo mai, solo perché qui è usato!
\ifclassica%
{\begin{dedica}
A non so fare una dedica

% \textdagger\ A non so
\end{dedica}
\fi
%%%%%%% Fine esperimento

% Si veda la documentazione per verificare la differenza fra abstract
% e summary. Perciò se se ne usa uno, non si deve usare l'altro.
% \english
% \begin{abstract}
% This short abstract, is typeset with the \texttt{abstract} environment (from the \texttt{report} document class) just to test if it works. or what concerns working, it works, but in Italian ist title turns out to be ``Sommario'' in bold face series and normal size; its apperance looks like  a bad copy of what one obtains with the \texttt{\string\summary} command. In English, though, its title is “Abstract”, as it should be, since at the beginning of this template a suitable \texttt{\string\ExtendCapions} command was issued.

% Please, read the documentation  in Italian (file \texttt{toptesi-it.pdf} in order to fully understand the difference beteesn “abstract” and “summary” in the context of this bundle.

% \end{abstract}

% Fine dell'altro esperimento
\italiano
\sommario
% todo: anomaly detection o anomaly-detection
Gli attacchi di denial of Service distribuiti (DDoS) sono uno dei maggiori problemi di sicurezza delle reti. Hanno lo scopo di impedire ad utenti legittimi l'accesso a dei servizi o degradare loro le prestazioni. Contestualmente, gli strumenti a disposizione di chi deve mitigare questo tipo di attacchi evolvono con l'affinamento delle tecniche di riconoscimento del traffico e con la capacità di monitorare le reti aziendali esposte su Internet. L'applicazione di tecniche di anomaly detection e l'utilizzo di software open source per il machine learning ha permesso di sviluppare un software in grado di individuare anomalie nel traffico di rete, con vantaggi prestazionali rispetto ai classici sistemi di intrusion detection e vantaggi economici rispetto alle soluzioni proprietarie dei vendor. Inoltre la questione di ottenere sempre maggiore padronanza della propria infrastruttura IT si scontra continuamente con l'aumentare del numero e con la continua evoluzione degli applicativi, della tecnologia e dell'espansione complessiva della rete Internet stessa. In questa tesi proveremo a identificare anomalie riconducibili ad attacchi DDoS, in un contesto di una rete aziendale con più sedi, usando un riconoscimento delle anomalie effettuato tramite  una rete neurale allenata su dati provenienti dai router di più sedi aziendali e una successiva mitigazione degli attacchi tramite un agent sugli stessi. Approfondiremo l'uso di una particolare tipologia di rete neurale chiamata autoencorder, che si sta diffondendo rapidamente sia in ambito accademico che industriale nell'ambito dell'anomaly-detection. Esporremmo il nostro contributo prendendo come caso d'uso dati raccolti da una sede distaccata della società Tiesse s.p.a., produttrice di router e altri dispositivi di rete, coinvolta nello studio e nella realizzazione di questo progetto.


% \paginavuota % funziona anche senza specificare l'opzione classica
\italiano

% \tablespagetrue\figurespagetrue % normalmente questa riga non serve ed e' commentata
\indici


\mainmatter
\chapter{Introduzione}


\section{Motivazione}
Prova 


\section{Gli attacchi DDoS}
Prova prova
\subsection{Tipologia di attacchi DDoS}
\subsection{Vittime attacchi DDoS}
\subsection{Diffusione attacchi DDoS}


\section{Organizzazione della tesi}

\chapter{Stato dell'arte}

\section{Riconoscimento DDoS}

\section{Contromisure attacchi DDoS}

\subsection{Soluzioni alla sorgente}

\subsection{Soluzioni alla destinazione}

\subsection{Soluzioni distribuite}

\chapter{Stato dell'arte dell'anomaly detecion - trovare titolo}

\section{Cos'è l'anomaly dedtection?}


L'Anomaly detection si riferisce al problema di trovare pattern nei dati che non sono conformi al comportamento aspettato. A queste non conformità ci si riferisce come anomalie \cite{anomaly_detection_survey_3}. L'importanza del rilevamento delle anomalie è il fatto che un'anomalia nei dati spesso corrisponde ad un'informazione critica nel dominio a cui si riferisce, per esempio nelle reti di computer un traffico anomalo potrebbe significare che un computer è stato hackerato e sta compiendo azioni per il danneggiamento dell'azienda.

\section{Sfide dell'anomaly detection}

Le anomalie sono definite come un pattern che non rispetta il normale comportamento, ma il definire il concetto di normalità è una sfida, i maggiori fattori che influiscono su questa decisione sono \cite{anomaly_detection_survey_3}:

\begin{itemize}
    \item La difficoltà nel trovare una regione che comprenda tutti i possibili comportamenti normali è molto difficile e il confine tra azioni normali e anomale spesso non è ben definito.
    \item Se le azioni anomale sono generate da azioni malevole, il responsabile cercherà di fare in modo che le osservazioni sui dati appaiano normali.
    \item In alcuni contesti il comportamento si evolve e ciò che è considerato correntemente normale potrebbe essere rappresentativo per il futuro.
    \item È difficile definire quanto la differenza dalla normalità debba essere considerata anomala, per esempio in medicina piccole variazioni per esempio della temperatura corporea possono essere considerate anormali, in finanza la fluttuazione del valore delle azioni potrebbe essere considerato normale.
    \item La disponibilità di dati già classificati come normali o anomali per verificare il modello è uno dei problemi principali.
    \item Spesso i dati contengono rumore che tende ad essere simile alle anomalie ed è difficile rimuoverlo o distinguerlo.
\end{itemize}


\section{Classificazione delle anomalie}

\subsection{Tipologia di anomalie}
Un aspetto importante dell'anomaly detection è l'analisi delle anomalie che possono presentarsi, di conseguenza le anomalie possono essere classificate nel seguente modo:
\begin{itemize}
    \item Anomalie puntuali: un singolo dato che si discosta dalla normalità. Questo è il caso più semplice e su cui si concentra la maggior parte delle ricerche sui dati anomali. Un esempio è un utente che tutti i giorni scarica 1GB di dati quando arriva in ufficio, ma un giorno ne scarica 10.
    \item Anomalie contestuali: quando un insieme di dati si comporta in modo anomalo in un determinato contesto, per esempio il numero di acquisti su un sito durante il periodo di Natale è più alto che durante il resto dell'anno.
    \item Anomalie collettive: quando un'istanza di dati è anormale rispetto all'intero dataset, in questo caso i dati in sè non sono anomali, ma lo diventano quando presi insieme, un esempio è l'elettrocardiogramma, in cui se ci sono bassi valori per un lungo periodo possono identificare un problema.
\end{itemize}

\subsection{Applicazione anomaly detection}
% \subsection{Tipologie anomaly detection}
% \begin{itemize}
%     \item Intrusion detection
%     \item Fraud detection
%     \item Medical and Public Health Anomaly Detection
%     \item 
% \end{itemize}

\subsection{Tipologia degli attacchi di rete}
%parlare di Network IDS E Host IDS pagina 9 \cite{anomaly_detection_survey_2_deep_learning}
\section{Sistemi di rilevamento delle anomalie}

\subsection{Metodo di detection}
% \cite{anomaly_detection_survey_1_network} pagina 24
\begin{itemize}
    \item Classification based
    \item Statistical anomaly detection
    \item Information theory
    \item Clustering-based
\end{itemize}

\subsection{Dati in input}
pagina 6 \cite{anomaly_detection_survey_3}

\subsection{Output of Anomaly Detection}

Etichette o Anomaly score



\section{Motivazione}
Prova 


\section{Reti neurali e funzionamento}



\subsection{Autoencoders}


\chapter{Il mio lavoro}


\section{Selezione features}
\subsection{Collectd}
\subsection{NDPI}

\section{Il mio tool}
\subsection{Struttura}
\subsection{Modello della rete}
\subsection{Train}
\subsection{Evaluate}


\section{Test sulle anomalie}
\subsection{Tool utilizzati}

Parlare come funzionano i tool che ho fatto

\subsection{Risultati}

\chapter{Mitigazione degli attacchi}


\section{Introduzione}
Prova 
\subsection{Bloccare l'ip spoofing}
L'ip spoofing permettere di usare la tecnica dell'amplification


\section{Funzionamento}
Prova prova
\subsection{eBPF}
\subsection{BCC}

Qua posso parlare di due alternative, la prima è riutilizzare il sistema di anomaly detection simile a quello presentato precedentemente elencando tutti i problemi e i vantaggi.

Il secondo consiste nel raccogliere i dati come prima, e creare un ranking per ogni feature risultata anomala precedentemente e a quel punto blocco i flussi sopra una certa soglia, ma quale?

Mentre per l'ip spoofing come la gestisco?

\section{Test sulle anomalie}
\subsection{Tool utilizzati}
\subsection{Risultati}

\chapter{Lavoro futuro}

% Ciao \cite{dirac}
\chapter{Conclusioni}

Sicuramente una cosa da migliorare è la sicurezza per assicurarsi che neanche le sedi periferiche non siano infettate


\chapter{Stato dell'arte}

\section{Gli attacchi DDoS}
Prova prova

\section{I satelliti medicei}
Prova prova

\chapter{Il barometro}
\section{Generalit\`a}
\begin{interlinea}{0.87} Il barometro, come dice il nome, serve per
misurare la pesantezza; pi\`u precisamente la pesantezza dell'aria
riferita all'unit\`a di superficie.
\end{interlinea}

\begin{interlinea}{2} Studiando il fenomeno fisico si pu\`o concludere
che in un dato punto grava il peso della colonna d'aria che lo
sovrasta, e che tale colonna \`e tanto pi\`u grave quanto maggiore
\`e la superficie della sua base; il rapporto fra il peso e la base
della colonna si chiama pressione e si misura in once toscane al cubito
quadrato, \cite{tor1}; nel Ducato di Savoia la misura in once al piede
quadrato \`e quasi uguale, perch\'e col\`a usano un piede molto
grande, che \`e simile al nostro cubito.
\end{interlinea}

\subsection{Forma del barometro}
Il barometro consta di un tubo di vetro chiuso ad una estremit\`a e
ripieno di mercurio, capovolto su di un vaso anch'esso ripieno di
mercurio; mediante un'asta graduata si pu\`o misurare la distanza fra
il menisco del mercurio dentro il tubo e la superficie del mercurio
dentro il vaso; tale distanza \`e normalmente di 10 pollici toscani,
\cite{tor1,tor2}, ma la misura pu\`o variare se si usano dei pollici
diversi; \`e noto infatti che gl'huomini sogliono avere mani di
diverse grandezze, talch\'e anche li pollici non sono egualmente
lunghi.
\section{Del mercurio}
Il mercurio \`e un a sostanza che si presenta come un liquido, ma ha il colore
del metallo. Esso \`e pesantissimo, tanto che un bicchiere, che se fosse pieno
d'acqua, sarebbe assai leggiero, quando invece fosse ripieno di mercurio,
sarebbe tanto pesante che con entrambe le mani esso necessiterebbe di essere
levato in suso.

Esso mercurio non trovasi in natura nello stato nel quale \`e d'uopo che sia
per la costruzione dei barometri, almeno non trovasi cos\`i abbondante come
sarebbe necessario.

\setcounter{footnote}{25}

Il Monte Amiata, che \`e locato nel territorio del Ducato%
\footnote{Naturalmente stiamo parlando del Granducato di Toscana.%
\ifclassica\NoteWhiteLine\fi
} del nostro Eccellentissimo et Illustrissimo Signore Granduca di Toscana\footnote{Cosimo IV de' Medici.}, \`e uno dei
luoghi della terra dove pu\`o rinvenirsi in gran copia un sale rosso, che
nomasi \emph{cinabro}, dal quale con artifizi alchemici, si estrae il mercurio
nella forma e nella consistenza che occorre per la costruzione del barometro
terrestre%
\ifclassica
\nota{Nota senza numero\dots

\dots e che va a capo.
}\fi.


La densit\`a del mercurio \`e molto alta e varia con la temperatura come
pu\`o desumersi dalla tabella \ref{t:1}.


Il mercurio gode della sorprendente qualit\`a et propriet\`a, cio\`e che esso
diventa tanto solido da potersene fare una testa di martello et infiggere
chiodi aguzzi nel legname.
\begin{table}[htp]              % crea un floating body col nome Tabella nella
                            % didascalia
\centering                      % comando necessario per centrare la tabella
\begin{tabular}%                % inizio vero e proprio della tabella
{rrrrrr}                        % parametri di incolonnamento
\hline\hline                    % filetti orizzontali sopra la tabella
                            % intestazione della tabella
\multicolumn{3}{c}{\rule{0pt}{2.5ex}Temperatura} % \rule serve per lasciare
& \multicolumn{3}{c}{Densit\`a} \\               % un po' di spazio sopra le parole
&\unit{\gradi C} & & & $\unit{t/m^3}$ &  \\
\hline%                         % Filetto orizzontale per separare l'intestazione
\hspace*{1.3em}& 0  &  & & 13,8 &  \\   % I numeri sono incolonnati % 
          & 10  &  & & 13,6 &  \\   % a destra; le colonne vuote
          & 50  &  & & 13,5 &  \\   % servono per centrare le colonne
          &100  &  & & 13,3 &  \\   % numeriche sotto le intestazioni
\hline \hline                           % Filetti di fine tabella
\end{tabular}
\caption[Densit\`a del mercurio]{Densit\`a del mercurio. Si pu\`o fare molto meglio usando il pacchetto \textsf{booktabs}.} \label{t:1}  % didascalia con label
\end{table}

%\selectlanguage{italian}

\begin{osservazione}\normalfont
Questa propriet\`a si manifesta quando esso \`e estremamente freddo, come
quando lo si immerge nella salamoia di sale e ghiaccio che usano li maestri
siciliani per confetionare li sorbetti, dei quali sono insuperabili artisti.
\end{osservazione}

Per nostra fortuna, questo grande freddo, che necessita per la confetione de
li sorbetti, molto raramente, se non mai, viene a formarsi nelle terre del
Granduca Eccellentissimo, sicch\'e non vi ha tema che il barometro di mercurio
possa essere ruinato dal grande gelo e non indichi la pressione giusta, come
invece deve sempre fare uno strumento di misura, quale \`e quello che \`e
descritto cost\`i.\cite{duane1964}

\chapter*{Conlusioni}
E con questo si conclude la tesi d'esempio per una tesi magistrale con un capitolo non  numerato che si trova ancora nella main matter.

Dovrebbe essere evidente che il comando \texttt{\string\chapter*} non dovrebbe mai essere usato nella main matter, tranne eventualmente un capitoletto conclusivo e riassuntivo \emph{non strutturato}. Infatti se esso contenesse al suo interno paragrafi, sottoparagrafi e affini, questi verrebbero numerati erroneamente con il numero del capitolo precedente. 

\appendix

\chapter{Il listato del pacchetto \texttt{topcoman.sty}}
\listing{topcoman.sty}

\chapter{Seconda appendice}
Questa è la seconda appendice numerata con una lettera perché questo comando \texttt{\string\chapter} viene dopo il comando \texttt{\string\appendix}.

Le appendici vanno numerate se sono più di una e devono quindi stare nella main matter, perché nella back matter nulla viene numerato.

La bibliografia che segue non è numerata perché l'ambiente \texttt{thebibliography} compone il suo titolo con il comando \texttt{\string\chapter*} e ne manda il titolo nell'indice generale con i suoi propri comandi interni. La definizione di questo ambiente è specifica di questo macro-pacchetto \texttt{TOPtesi}.

Sarebbe meglio inserire la bibliografia dopo un comando \texttt{\string\backmatter} esplicito. La back matter è destina espressamente a una sola appendice non numerata (ci pensa da sola a non numerare le sue sezioni); alla bibliografia, a uno o più indici analitici, a glossari o nomenclature, liste di acronimi, e simili. Nulla è obbligatorio in una back matter, ma una tesi senza bibliografia non sarebbe appropriata, tanto meno una tesi magistrale priva di una bibliografia.

\begin{thebibliography}{9}
    \bibitem{ddos_survey_1} Saman Taghavi Zargar, James Joshi and David Tipper {\em A Survey of Defense Mechanisms Against Distributed Denial of Service (DDoS) Flooding Attacks}, 2013. ()
s
    \bibitem{ddos_survey_2} s

    \bibitem{ddos_motivations} Tasnuva Mahjabin, Yang Xiao, Guang Sun and Wangdong Jiang {\em A survey of distributed denial-of-service attack, prevention, and mitigation techniques}, 2017. (https://journals.sagepub.com/doi/10.1177/1550147717741463)

    \bibitem{ddos_survey_3} Neha Agrawal and Shashikala Tapaswi {\em Defense schemes for variants of distributed denial-of-service (DDoS) attacks in cloud
    computing: A survey}, 2017. (https://doi.org/10.1080/19393555.2017.1282995)

    \bibitem{ddos_survey_4} Jasmeen Kaur Chahal, Abhinav Bhandari and Sunny Behal {\em Distributed Denial of Service Attacks: A Threat or Challenge}, 2019. (https://doi.org/10.1080/13614576.2019.1611468)

    \bibitem{ddos_kaspersky} kaspersky securitylist.com {\em DDoS Report - DDoS attacks in Q4 2020
    }, 2020. (https://securelist.com/ddos-attacks-in-q4-2020/100650/)
    
    \bibitem{ddos_kaspersky_q3_2020} kaspersky securitylist.com {\em DDoS Report - DDoS attacks in Q3 2020
    }, 2020. (https://securelist.com/ddos-attacks-in-q4-2020/100650/)

    %% da qui va cancellata
    % \bibitem{gal} G.~Galilei, {\em Nuovi studii sugli astri medicei}, Manuzio, Venetia, 1612.

    % \bibitem{tor1} E.~Torricelli, in ``La pressione barometrica'', {\em Strumenti

    %         Moderni}, Il Porcellino, Firenze, 1606.
    % \bibitem{tor2} E.~Torricelli e A.~Vasari, in ``Delle misure'', {\em Atti Nuovo
    %         Cimento}, vol.~III, n.~2 (feb. 1607), p.~27--31.

    % \bibitem{duane1964} Duane J.T., \emph{Learning Curve Approach To Reliability Monitoring}, IEEE Transactions on Aerospace, Vol. 2, pp. 563-566, 1994

    % % altri riferimenti da usare come esempi.

    % \bibitem{chiesa2008} Chiesa S., \emph{Affidabilità, sicurezza e manutenzione
    %     nel progetto dei sistemi}, CLUT, gennaio 2008
    % \bibitem{chiesa2}Chiesa S., Fioriti M., Fusaro R., \emph{On Board System
    %     Technological  Level Improvement Effect on UAV MALE}
    % \bibitem{bigliano2010} Bigliano M., \emph{Sicurezza nell'installazione di un velivolo
    %     senza pilota MALE; applicazione di metodologia di Zonal Safety
    %     Analysis al velivolo del Progetto SAvE}, Politecnico di Torino,
    % maggio 2010
    % \bibitem{astrid2012} Chiesa S., Di Meo G.A., Fioriti M., Medici G., Viola N.,
    % \emph{ASTRID - Aircraft on board Systems sizing and TRade-off
    %     analysis in Initial Design}, Research Bulletin, Warsaw University
    % of Technology, Institute of Aeronautics and Applied Mechanics,
    % p. 1-28, 17-19, ottobre 2012

\end{thebibliography}




\end{document}

