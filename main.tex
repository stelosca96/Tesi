% !TEX encoding = UTF-8 Unicode
% !TEX TS-program = pdflatex

%%%%%%% La riga soprastante serve per configurare gli editor
%%%%%%% TeXShop, TeXworks e TeXstudio per gestire questo file
%%%%%%% con la codifica UFF-8.
%%%%%%% Se si vuole usare un'altra codifica si veda sotto.
%%%%%%%

%%%%%%%  Esempio con molte opzioni
%%%%%%% Le opzioni nella forma "chiave=valore" sono definite
%%%%%%% perché la classe dalla versione 6.1.00 usa il pacchetto
%%%%%%% xkeyval. Vedere sulla documentazione in inglese o
%%%%%%% in italiano quali chiavi accettano valori.

%%%%%%% L'opzione per il corpo accetta qualsiasi valore, anche fratto
%%%%%%% (per esempio: corpo=11.5pt) e va sempre scritto con una
%%%%%%% unità di misura. L'utente è pregato di non esagerare con
%%%%%%% corpi normali minori di 9.5pt o maggiori di 13pt.
%%%%%%%
%%%%%%% Le opzioni per inputenc e fontenc vanno per prime.
%%%%%%% Vengono ignorate se NON si compone con pdfLaTeX. Ma
%%%%%%% questo è un esempio per pdfLaTeX.
%%%%%%%

\documentclass[%
corpo=12pt,
twoside,
stile=standard,
% oldstyle,
%    autoretitolo,
tipotesi=magistrale,
numerazioneromana,
% libro, utile per la stampa => mette il margine diverso per la rilegatura
greek,
evenboxes,
]{toptesi}
%%%%%%%%%%%%%%%%%%%%%%%%%%%%%%%%%%%%%%%%%%%%%%%%%%%%
%%%%%% Per la codifica d'entratasi può scegliere quella che si vuole,
%%%%%% ma si consiglia di preferire utf8; in ogni caso non scegliere
%%%%%% codifiche specifiche del sistema operativo.

\usepackage[utf8]{inputenc}% codifica d'entrata
\usepackage[T1]{fontenc}%    codifica dei font
\usepackage{lmodern}%        scelta dei font
% \usepackage[
% backend=biber,
% style=alphabetic,
% sorting=ynt
% ]{biblatex}
% \addbibresource{biblio.bib}
% Vedere la documentazione toptesi-it.pdf per le
% attenzioni che bisogna usare al fine di ottenere un file
% veramente conforme alle norme per l'archiviabilità.
\usepackage[style=numeric-comp,useprefix,hyperref,backend=bibtex]{biblatex}
\bibliography{biblio.bib}


\usepackage{amsmath}
\usepackage{dirtree}
\usepackage{listings}             % Include the listings-package per il codice
%%% codice per supportare i json
\usepackage{xcolor}

% sottolineature che vanno a capo
\usepackage[normalem]{ulem}
\usepackage{tabularx}


\colorlet{punct}{red!60!black}
\definecolor{background}{HTML}{EEEEEE}
\definecolor{delim}{RGB}{20,105,176}
\colorlet{numb}{magenta!60!black}

\lstdefinelanguage{json}{
    % basicstyle=\normalfont\ttfamily,
    basicstyle=\footnotesize\ttfamily\linespread{0},
    numbers=left,
    numberstyle=\scriptsize,
    % stepnumber=1,
    % numbersep=8pt,
    showstringspaces=false,
    breaklines=true,
    % frame=lines,
 %   backgroundcolor=\color{background},
    literate=
     *{0}{{{\color{numb}0}}}{1}
      {1}{{{\color{numb}1}}}{1}
      {2}{{{\color{numb}2}}}{1}
      {3}{{{\color{numb}3}}}{1}
      {4}{{{\color{numb}4}}}{1}
      {5}{{{\color{numb}5}}}{1}
      {6}{{{\color{numb}6}}}{1}
      {7}{{{\color{numb}7}}}{1}
      {8}{{{\color{numb}8}}}{1}
      {9}{{{\color{numb}9}}}{1}
      {:}{{{\color{punct}{:}}}}{1}
      {,}{{{\color{punct}{,}}}}{1}
      {\{}{{{\color{delim}{\{}}}}{1}
      {\}}{{{\color{delim}{\}}}}}{1}
      {[}{{{\color{delim}{[}}}}{1}
      {]}{{{\color{delim}{]}}}}{1},
}
%% fine codice per supportare i jsons

\lstdefinelanguage{python3}{
    % basicstyle=\normalfont\ttfamily,
    basicstyle=\footnotesize\ttfamily\linespread{0},
    numbers=left,
    numberstyle=\scriptsize,
    % stepnumber=1,
    % numbersep=8pt,
    showstringspaces=false,
    breaklines=true,
    % frame=lines,
 %   backgroundcolor=\color{background},
    literate=
     *{0}{{{\color{numb}0}}}{1}
      {1}{{{\color{numb}1}}}{1}
      {2}{{{\color{numb}2}}}{1}
      {3}{{{\color{numb}3}}}{1}
      {4}{{{\color{numb}4}}}{1}
      {5}{{{\color{numb}5}}}{1}
      {6}{{{\color{numb}6}}}{1}
      {7}{{{\color{numb}7}}}{1}
      {8}{{{\color{numb}8}}}{1}
      {9}{{{\color{numb}9}}}{1}
      {:}{{{\color{punct}{:}}}}{1}
      {,}{{{\color{punct}{,}}}}{1}
      {\{}{{{\color{delim}{\{}}}}{1}
      {\}}{{{\color{delim}{\}}}}}{1}
      {[}{{{\color{delim}{[}}}}{1}
      {]}{{{\color{delim}{]}}}}{1},
}
%% fine codice per supportare i json

\usepackage{hyperref}

\hypersetup{%
    pdfpagemode={UseOutlines},
    bookmarksopen,
    pdfstartview={FitH},
    colorlinks,
    linkcolor={blue},
    citecolor={blue},
    urlcolor={blue}
}
%
%%%%%%% Esempio di composizione di tesi di laurea con PDFLATEX 
%
%
% Per scrivere testo fasullo in "latinorum"
\usepackage{lipsum}
%

%%%%%%% Definizioni locali
\newtheorem{osservazione}{Osservazione}% Standard LaTeX
\ExtendCaptions{italian}{Abstract}{Acknowledgements}

\begin{document}\errorcontextlines=9
%%%%%%% Questi comandi è meglio metterli dentro l'ambiente
%%%%%%% ThesisTitlePage con o senza asterisco, oppure in un file di
%%%%%%% configurazione personale. Si veda la documentazione
%%%%%%% inglese o italiana.
%%%%%%% Comunque i presenti comandi servono per comporre la
%%%%%%% tesi con i moduli di estensione standard del pacchetto
%%%%%%% TOPtesi.

\begin{frontespizio*}
    % Per cambiare la dicitura sopra la lista dei laureandi decommentare
    % la riga seguente, cambiando le 4 parole in modo consistente
    %

    \TitoloListaCandidati{Studente,Studenti,Studentessa,Studentesse}
    %
    \ateneo{Politecnico di Torino}
    \logosede[150px]{polito_logo_2021_blu}% 

    %
    % Non tutte le università hanno un nome proprio
    % \nomeateneo{Sede di Torre sosjd}
    %
    % \struttura[III]{Matematica, Fisica e~Scienze Naturali}
    %\Materia{Remote sensing}
    
    \titolo{Anomaly detection per il rilevamento di attacchi DDoS su reti aziendali}% per la laurea quinquennale e il dottorato
    % \sottotitolo{Metodo dei satelliti medicei}% per la laurea quinquennale e il dottorato
    %
    %%%%%%% Corso degli studi
    \corsodilaurea{Ingegneria Informatica}% per la laurea
    
    %%%%%%% L'eventuale numero di matricola va fra parentesi quadre
    % \show\Candidato
    \def\Candidato{Candidato}
    %\show\Candidato
    \candidato{Stefano \textsc{Loscalzo}}[s267614] 
    %\secondocandidato{Evangelista \textsc{Torricelli}}[123457]
    
    %%%%%%% Relatori o supervisori
    %
    \relatore{prof.~Guido \textsc{Marchetto}}
    % \secondorelatore{dipl.~ing.~Werner von Braun}
    % \terzorelatore{dott.~Mario Rossi}
    % 
    %%%%%%% Per mettere altri relatori consultare toptesi-it.pdf
    
    %%%%%%% Tutore
    \tutoreaziendale{ing.\ Francesco \textsc{Lucrezia}}
    \NomeTutoreAziendale{Supervisore aziendale\\Tiesse S.p.A.}
    
    %%%%%%% Seduta dell'esame
    %\sedutadilaurea{Agosto 1615}
    %%%%%%%% oppure:
    \sedutadilaurea{Anno~Accademico 2020-2021}% 
    
    %%%%%%% Logo della sede
\end{frontespizio*}


%%%%%%% Per cambiare l'offset per la rilegatura;
%%%%%%% meno offset c'e', meglio e'
%\setbindingcorrection{3mm}




%%%%%%% Questo test è usato appunto per collaudare diversi stili,
%%%%%%% non per comporre una vera tesi.
%%%%%%% Non usarlo mai, solo perché qui è usato!
\ifclassica%
{\begin{dedica}
Dedica qui
% \textdagger\ A non so
\end{dedica}
\fi
%%%%%%% Fine esperimento

% Si veda la documentazione per verificare la differenza fra abstract
% e summary. Perciò se se ne usa uno, non si deve usare l'altro.
% \english
% \begin{abstract}
% This short abstract, is typeset with the \texttt{abstract} environment (from the \texttt{report} document class) just to test if it works. or what concerns working, it works, but in Italian ist title turns out to be ``Sommario'' in bold face series and normal size; its apperance looks like  a bad copy of what one obtains with the \texttt{\string\summary} command. In English, though, its title is “Abstract”, as it should be, since at the beginning of this template a suitable \texttt{\string\ExtendCapions} command was issued.

% Please, read the documentation  in Italian (file \texttt{toptesi-it.pdf} in order to fully understand the difference beteesn “abstract” and “summary” in the context of this bundle.

% \end{abstract}

% Fine dell'altro esperimento
\italiano
\sommario
% todo: anomaly detection o anomaly-detection
Gli attacchi di denial of Service distribuiti (DDoS) sono uno dei maggiori problemi di sicurezza delle reti. Hanno lo scopo di impedire ad utenti legittimi l'accesso a dei servizi o degradare loro le prestazioni. Contestualmente, gli strumenti a disposizione di chi deve mitigare questo tipo di attacchi evolvono con l'affinamento delle tecniche di riconoscimento del traffico e con la capacità di monitorare le reti aziendali esposte su Internet. L'applicazione di tecniche di anomaly detection e l'utilizzo di software open source per il machine learning ha permesso di sviluppare un software in grado di individuare anomalie nel traffico di rete, con vantaggi prestazionali rispetto ai classici sistemi di intrusion detection e vantaggi economici rispetto alle soluzioni proprietarie dei vendor. Inoltre la questione di ottenere sempre maggiore padronanza della propria infrastruttura IT si scontra continuamente con l'aumentare del numero e con la continua evoluzione degli applicativi, della tecnologia e dell'espansione complessiva della rete Internet stessa. In questa tesi proveremo a identificare anomalie riconducibili ad attacchi DDoS, in un contesto di una rete aziendale con più sedi, usando un riconoscimento delle anomalie effettuato tramite  una rete neurale allenata su dati provenienti dai router di più sedi aziendali e una successiva mitigazione degli attacchi tramite un agent sugli stessi. Approfondiremo l'uso di una particolare tipologia di rete neurale chiamata autoencorder, che si sta diffondendo rapidamente sia in ambito accademico che industriale nell'ambito dell'anomaly-detection. Esporremmo il nostro contributo prendendo come caso d'uso dati raccolti da una sede distaccata della società Tiesse s.p.a., produttrice di router e altri dispositivi di rete, coinvolta nello studio e nella realizzazione di questo progetto.


% \paginavuota % funziona anche senza specificare l'opzione classica
\italiano

% \tablespagetrue\figurespagetrue % normalmente questa riga non serve ed e' commentata
\indici


\mainmatter
\chapter{Introduzione}


\section{Motivazione}
Prova 


\section{Gli attacchi DDoS}
Prova prova
\subsection{Tipologia di attacchi DDoS}
\subsection{Vittime attacchi DDoS}
\subsection{Diffusione attacchi DDoS}


\section{Organizzazione della tesi}


\chapter{Sistemi anti-DDoS: stato dell'arte}
% attacchi e contromisure sui ddos

% Quando Internet è stato inventato non si pensava a questo problema e non sono state implementate difese

\section{Riconoscimento DDoS}

% \cite{ddos_motivations} pagina 16, \cite{ddos_survey_4} pagina 66
La fase di riconoscimento degli attacchi DDoS è un importante passo per permettere la successiva mitigazione, questa fase diventa più facile maggiormente ci avviciniamo alla vittima dell'attacco, ma viceversa più ci si allontana dalla sorgente dell'attacco e più diventa difficile identificarla. In letteratura esistono due tecniche per identificare i flussi malevoli: signature-based detection e anomaly-based detection.

\subsection{Signature-based detection}

La signature-based detection è un meccanismo che si basa su attacchi DDoS conosciuti, differenziando la loro firma dai normali flussi della rete. Queste soluzioni hanno un buon successo con attacchi DDoS noti, ma non sono in grado di rilevare nuove tipologie di attacco delle quali non si conosce ancora la signature. Questi sistemi si possono basare su pattern matching (es. Bro/Zeek), su regole (es. Snort, Suricata), sulla correlazione di informazioni di management sul traffico o sull'analisi spettrale.
% todo: rivedere le ultime due perché non so bene cosa siano

\subsection{Anomaly-based detection}

I sistemi basati sul rilevamento delle anomalie possono riconoscere anche attacchi non conosciuti, basandosi su delle soglie per differenziare il traffico normale da quello malevolo, ma la scelta dei limiti oltre i quali considerare il traffico anomalo è una grande sfida per questa tipologia di tecniche.
I metodi più diffusi si basano su metodi statistici, di data mining o intelligenze artificiali.

\section{Contromisure attacchi DDoS}

Gli attacchi DDoS si raggruppano ad imbuto dalle sorgenti verso la vittima, per questa ragione più si è vicini alla vittima e più l'attacco sarà facile da riconoscere, ma più difficile da mitigare. Per questa ragione le tecniche di mitigazione vengono suddivise in base al luogo in cui vengono azionate.

\begin{figure}[]
    % todo: capire come gestire citazioni imsmagini a livello di copyright
    %  e capire come funzionano le label per richiamare le immagini
    \label{fig:classificazione_ddos}
    \includegraphics[width=\hsize]{images/ddos/classificazione_difese.png}
    \caption{Classificazione dei sistemi di difesa in base al luogo di applicazione \cite{ddos_survey_1}}
    \centering
\end{figure}

\subsection{Soluzioni alla sorgente}
Questa tipologia di soluzioni sono adottate vicino alle sorgenti dell'attacco per impedire agli utenti di una sottorete di generare attacchi DDoS. Queste soluzioni possono essere applicate agli edge router degli Autonomous System (AS) di accesso.

Degli esempi di soluzioni sono \cite{ddos_survey_1}:

% todo: completare elenco delle soluzioni alla sorgente
\begin{itemize}
    \item Filtri in ingresso e uscita agli edge router delle sorgenti: gli attacchi che si basano sulla riflessione sfruttano la tecnica dell'ip spoofing, lo scopo di questi filtri è di bloccare il traffico che lo utilizza senza interferire sul traffico legittimo.
    \item D-WARD: sistema che mira a rilevare attacchi di tipo DDoS flooding monitorando il traffico in ingresso e uscita e comparandolo con dei flussi normali di traffico. I flussi sono filtrati se non rispettano il modello. Per esempio può essere imposto un rapporto massimo tra pacchetti inviati e ricevuti per un flusso TCP che se superato può segnalare il flusso come anomalo.
    \item MULTOPS (MUlti-Level Tree for Online Packets Statistics): è un'euristica e un struttura dati che permette di rilevare attacchi provenienti da una subnet. Normalmente il traffico in una direzione è proporzionale a quello in quella opposta, una significativa differenza possono indicare che la rete è la sorgente o la vittima di un attacco DDoS.
    \item MANAnet’s Reverse Firewall: è un firewall che a differenza di quelli tradizionali, protegge l'esterno dal traffico in uscita dalla subnet.
\end{itemize}

I problemi di questa soluzione sono che per permettere una copertura totale, questi meccanismi dovrebbe essere implementatati su tutti gli edge router di tutti gli AS di accesso, inoltre è difficile differenziare il traffico legittimo, da quello malevolo e non meno importante non è chiaro chi sia il responsabile del mantenimento economico di questo servizio \cite{ddos_survey_1}.

\subsection{Soluzioni alla destinazione}

Esistono soluzioni che si possono applicare agli edge router della vittima, possono analizzare il suo comportamento, il suo traffico usuale e riconoscere le anomalie \cite{ddos_survey_1,ddos_survey_2}.
Delle soluzioni posizionate in questi luoghi possono essere dei proxy, firewall che gestiscono le connessioni semi aperte in caso di syn flood oppure l'utilizzo sistemi di tracciamento implementati in alcuni router (in caso di ip spoofing), per risalire alla vera sorgente dell'attacco e bloccarla. % todo: non so come concludere le soluzioni alla destinazione, sicuramente c'è qualche meccanismo che ho dimenticato \cite{ddos_survey_2, ddos_survey_1}

% todo: potrei descriverli in maniera più dettagliata 

Questi sistemi di difesa possono diventare i target degli attacchi, poiché spesso richiedono una grande quantità di memoria e potenza di processamento per effettuare le osservazioni delle misure statistiche \cite{ddos_survey_4}.

\subsection{Soluzioni sulla rete}

I sistemi anti-DDoS sulla rete si basano sui router o su firewall installati sulla rete dell'operatore.
Una prima soluzione adottata è quella del Router based packet filter, la quale si basa sui criteri dell'ingress filtering, ma applicandola ai router nel core della rete. Il traffico per ogni link tendenzialmente viene generato da un ristretto intervalli di indirizzi ip, quando appare un indirizzo ip sospetto viene filtrato, questa soluzione è adatta a rilevare attacchi che utilizzano ip spoofing, ma è inutile nel caso di utilizzo di ip genuini.
Altre soluzioni mirano ad identificare i router nel core di internet che sono stati compromessi e si comportano in modo anomalo, oppure mirano all'installazione di detection systems (DSs) i quali permettono di rilevare pattern anomali, ma sono computazionalmente molto dispendiosi.

I problemi di questa soluzione è l'introduzione di un overhead per il processamento sui router, inoltre il rilevamento degli attacchi è difficile a causa dei dati troppo poco aggregati.

\subsection{Soluzioni distribuite}

Le soluzioni distribuite creano una cooperazione tra i luoghi di installazione delle difese. Alla sorgente vengono installati i sistemi di difesa per filtrare l'attacco, mentre sulla rete della vittima viene effettuato il riconoscimento dell'attacco. Questa soluzione porta sia il vantaggio della facilità di riconoscimento degli attacchi possibile alla destinazione, sia l'efficienza dei sistemi per mitigare gli attacchi alla sorgente.
Gli svantaggi di questa soluzione sono la complessità e l'overhead dovuto alla comunicazione tra le componenti distribuite e il bisogno di comunicazioni/componenti fidati tra le varie parti responsabili della cooperazione.
Degli esempi di soluzioni distribuite sono i seguenti:
\begin{itemize}
    \item Hybrid packet marking and throttling/filtering mechanisms: sono meccanismi in cui viene rilevato l'attacco vicino alla vittima, la quale informa i router in upstream di limitare o filtrare quel traffico. Degli esempi di applicazione di questa tecnica sono:
    \begin{itemize}
        \item Aggregate-based  congestion Control (ACC and Pushback: rileva un sottogruppo di traffico aggregato secondo alcune caratteristiche e richiede ai router in upstream di limitare o filtrare quel traffico.
        \item Attack diagnosis (AD) and parallel-AD: la vittima attiva l'AD quando un attacco viene rilevato, mandando i comandi correlati ai router in upstream, che contrassegneranno i pacchetti in transito con l'interfaccia di provenienza. Completato il traceback fino alla sorgente verrà mandato il comando AD per filtrare i pacchetti in arrivo alla sorgente.
        % todo: track
        \item TRACK
    \end{itemize}
    \item DEFensive Cooperative Overlay Mesh (DEFCOM): è un framework distribuito per abilitare lo scambio di informazioni tra tutti i nodi di difesa. Questo sistema prova a passare da un'architettura di difesa isolata ad una distribuita.
    \item Capability-based mechanisms: questo meccanismo permette alla destinazione di autorizzare esplicitamente il traffico che vuole ricevere.
    \item StopIt: meccanismo distribuito che prevede un sistema di autenticazione: tramite il quale viene prevenuto l'ip spoofing, e a dei pacchetti costruiti per essere inoltrati ai router in upstream che permettono il filtraggio del traffico il più vicino possibile alla sorgente del traffico.

\end{itemize}

\section{Momenti in cui mettere in atto le difese}

Esistono più momenti in cui è opportuno mettere in atto le difese contro attacchi DDoS e un buon sistema deve agire su tutti.
Il primo momento è prima dell'attacco (attack prevention), in cui vengono installati sistemi di monitoring, sistemi da usare in caso di emergenza, meccanismi per tollerare l'attacco (firewall, IDS, filtri, load balancer e flow control, strategie per identificare gli utenti legittimi e protezioni lato server). Successivamente durante l'attacco devono essere lanciate le adeguate contromisure e dopo l'attacco il sistema di difesa deve identificare la fonte e provvedere con le opportune contromisure, anche collaborando con altri segnalare l'attacco per permettere di difendersi da successivi.

\begin{figure}[]
    \label{fig:albero_difese}
    \includegraphics[width=\hsize]{images/ddos/ddos_flooding_defence.png}
    \caption{Meccanismi di difesa contro DDoS di tipo flooding.}
    \centering
\end{figure}

% \section{Tolleranza}


\chapter{Stato dell'arte dell'anomaly detecion - trovare titolo}

\section{Cos'è l'anomaly dedtection?}


L'Anomaly detection si riferisce al problema di trovare pattern nei dati che non sono conformi al comportamento aspettato. A queste non conformità ci si riferisce come anomalie \cite{anomaly_detection_survey_3}. L'importanza del rilevamento delle anomalie è il fatto che un'anomalia nei dati spesso corrisponde ad un'informazione critica nel dominio a cui si riferisce, per esempio nelle reti di computer un traffico anomalo potrebbe significare che un computer è stato hackerato e sta compiendo azioni per il danneggiamento dell'azienda.

\section{Sfide dell'anomaly detection}

Le anomalie sono definite come un pattern che non rispetta il normale comportamento, ma il definire il concetto di normalità è una sfida, i maggiori fattori che influiscono su questa decisione sono \cite{anomaly_detection_survey_3}:

\begin{itemize}
    \item La difficoltà nel trovare una regione che comprenda tutti i possibili comportamenti normali è molto difficile e il confine tra azioni normali e anomale spesso non è ben definito.
    \item Se le azioni anomale sono generate da azioni malevole, il responsabile cercherà di fare in modo che le osservazioni sui dati appaiano normali.
    \item In alcuni contesti il comportamento si evolve e ciò che è considerato correntemente normale potrebbe essere rappresentativo per il futuro.
    \item È difficile definire quanto la differenza dalla normalità debba essere considerata anomala, per esempio in medicina piccole variazioni per esempio della temperatura corporea possono essere considerate anormali, in finanza la fluttuazione del valore delle azioni potrebbe essere considerato normale.
    \item La disponibilità di dati già classificati come normali o anomali per verificare il modello è uno dei problemi principali.
    \item Spesso i dati contengono rumore che tende ad essere simile alle anomalie ed è difficile rimuoverlo o distinguerlo.
\end{itemize}


\section{Classificazione delle anomalie}

\subsection{Tipologia di anomalie}
Un aspetto importante dell'anomaly detection è l'analisi delle anomalie che possono presentarsi, di conseguenza le anomalie possono essere classificate nel seguente modo:
\begin{itemize}
    \item Anomalie puntuali: un singolo dato che si discosta dalla normalità. Questo è il caso più semplice e su cui si concentra la maggior parte delle ricerche sui dati anomali. Un esempio è un utente che tutti i giorni scarica 1GB di dati quando arriva in ufficio, ma un giorno ne scarica 10.
    \item Anomalie contestuali: quando un insieme di dati si comporta in modo anomalo in un determinato contesto, per esempio il numero di acquisti su un sito durante il periodo di Natale è più alto che durante il resto dell'anno.
    \item Anomalie collettive: quando un'istanza di dati è anormale rispetto all'intero dataset, in questo caso i dati in sè non sono anomali, ma lo diventano quando presi insieme, un esempio è l'elettrocardiogramma, in cui se ci sono bassi valori per un lungo periodo possono identificare un problema.
\end{itemize}

\subsection{Applicazione anomaly detection}
% \subsection{Tipologie anomaly detection}
% \begin{itemize}
%     \item Intrusion detection
%     \item Fraud detection
%     \item Medical and Public Health Anomaly Detection
%     \item 
% \end{itemize}

\subsection{Tipologia degli attacchi di rete}
%parlare di Network IDS E Host IDS pagina 9 \cite{anomaly_detection_survey_2_deep_learning}
\section{Sistemi di rilevamento delle anomalie}

\subsection{Metodo di detection}
% \cite{anomaly_detection_survey_1_network} pagina 24
\begin{itemize}
    \item Classification based
    \item Statistical anomaly detection
    \item Information theory
    \item Clustering-based
\end{itemize}

\subsection{Dati in input}
pagina 6 \cite{anomaly_detection_survey_3}

\subsection{Output of Anomaly Detection}

Etichette o Anomaly score



\section{Motivazione}
Prova 


\section{Reti neurali e funzionamento}



\subsection{Autoencoders}


\chapter{Il mio lavoro}


\section{Selezione features}
\subsection{Collectd}
\subsection{NDPI}

\section{Il mio tool}
\subsection{Struttura}
\subsection{Modello della rete}
\subsection{Train}
\subsection{Evaluate}


\section{Test sulle anomalie}
\subsection{Tool utilizzati}

Parlare come funzionano i tool che ho fatto

\subsection{Risultati}

\chapter{Soluzione proposta - Titolo provvisorio}

\section{Introduzione}
%spiegare obiettivo
Il nostro obiettivo è di creare un sistema di anomaly detection con struttura distribuita: nei router delle sedi periferiche vengono raccolte informazioni sul traffico e in un server posizionato fisicamente nella sede centrale lo analizzo e prendo decisioni sulle azioni da compiere. Questa tipologia di architettura permette di sfruttare dispositivi già esistenti e non aggiunge molta complessità all'infrastruttura informatica aziendale,quindi per svolgere questo lavoro ci siamo basati su alcuni software e sull'hardware aziendale con i relativi vantaggi e svantaggi rispetto ad una soluzione completamente ex-novo.
% spiegare come vengono gestiti i dati
La raccolta e l'analisi dei dati viene effettuata utilizzando un sistema centralizzato, questo permette di avere un unico dispositivo con la capacità necessaria a svolgere l'elaborazione, che al tempo stesso possiede una visione completa sulle sorgenti e quindi sarà in grado di identificare facilmente da quale sede proviene il traffico anomalo.

\section{Obiettivo}

\subsection{Gestione dei dati}

I dati vengono collezionati tramite un demone sui router (collectd), inviati al server (go-graphite) e visualizzati tramite una dashboard(graphana). Il tool di anomaly detection da me pensato si interfaccia direttamente con il database per la lettura dei dati e la scrittura dei risultati.
Qui in seguito approfondirò maggiormente il flusso della gestione dei dati.

\begin{figure}[]
    % todo: capire come gestire citazioni imsmagini a livello di copyright
    %  e capire come funzionano le label per richiamare le immagini
    \label{fig:mos}
    \includegraphics[width=\hsize]{images/my_work/tiesse_mos.png}
    \caption{Architettura di MOS (Monitor System di Tiesse)}
    \centering
\end{figure}

\paragraph{collectd} è un demone che raccoglie metriche di sistema e di applicazioni, trasferisce e salva dati di computer e dispositivi di rete. Collect ha una struttura modulare in cui è possibile abilitare centinaia di plugin per la raccolta di metriche di sistema dai casi più generali a quelli più specifici ed inoltre è possibile scrivere i propri plugin per integrarlo ulteriormente. I plugin usati sono ``write\_graphite'': plugin che permette di scrivere le metriche raccolte su un database graphite, ``conntrack'': plugin che permette di contare il numero di voci nella connection tracking table di Linux, ``interface'' : plugin che colleziona informazioni sul traffico su un'interfaccia, quindi pacchetti al secondo, bytes al secondo ed errori sull'interfaccia. 

Inoltre per avere ulteriori dati a disposizione ho scritto un plugin che si occupa di aggiungere delle metriche sul conteggio dei pacchetti non possibile con i plugin standard.

\paragraph{Il nostro plugin di collect}
%todo: posso parlare della prima versione, dei problemi riscontrati e dell'utilizzo di ebpf al suo posto

\paragraph{NDPI} è un software per il deep-packet inspection basato su OpenDPI
%http://luca.ntop.org/nDPI.pdf
%todo: completare ndpi


\paragraph{graphite} è un software open source per il monitoraggio che può funzionare sia su hardware economico, che su un'infrastruttura cloud. Può essere usato per monitorare le performance di siti, applicazioni, server e nel caso di Tiesse è usato per monitorare informazioni sull'uso dei router, come per esempio il numero totale di router connessi, quelli raggiungibili, il throughput, l'uptime e la velocità della connessione xDSL.
L'obiettivo di graphite è il salvataggio di serie temporali di dati numerici e la successiva condivisione e visualizzazione.
Graphite è composto da tre parti (come si può vedere dalla figura ~\ref{fig:graphite}):

% https://www.aosabook.org/en/graphite.html
% https://graphiteapp.org/#gettingStarted
%todo: spiegare meglio come funzionano le carbon api
\begin{itemize}
    \item carbon: è un service ad altre prestazioni che si occupa di ricevere le metriche con formato ``(timestamp, value)'' da salvare.
    \item whisper: un semplice database che salva sul filesystem le sequenze temporali di dati.
    \item graphite-web: è un'interfaccia utente e delle API le quali restituiscono i dati per renderizzare i grafici da visualizzare.
\end{itemize}

\begin{figure}[h]
    % todo: capire come gestire citazioni imsmagini a livello di copyright
    %  e capire come funzionano le label per richiamare le immagini
    \label{fig:graphite}
    \includegraphics[width=\hsize]{images/my_work/graphite.png}
    \caption{Architettura di Graphite}
    \centering
\end{figure}

\paragraph{go-graphite} è un'implementazione in Golang di Graphite, rispetto alla versione originale di carbon il backend
go-carbon è più veloce in tutte le condizioni (la differenza di prestazioni varia in base alla macchina su cui è installato) ~\ref{fig:gocarbon}.
Le carbonapi invece sono un subset delle api di graphite-web e le vanno a sostituire essendo dalle 5 alle 10 volte più veloci.

\begin{figure}[h]
    % todo: capire come gestire citazioni imsmagini a livello di copyright
    %  e capire come funzionano le label per richiamare le immagini
    \label{fig:gocarbon}
    \includegraphics[width=\hsize]{images/my_work/go-carbon.png}
    \caption{Differenza di prestazioni tra carbon e go-carbon}
    \centering
\end{figure}
% https://github.com/go-graphite/go-carbon


\begin{figure}[h]
    % todo: capire come gestire citazioni imsmagini a livello di copyright
    %  e capire come funzionano le label per richiamare le immagini
    \label{fig:graphana}
    \includegraphics[width=\hsize]{images/my_work/grafana_dashboard.png}
    \caption{La mia dashboard su Grafana}
    \centering
\end{figure}

\paragraph{Grafana} è un software open source che permette la visualizzazione e la generazione delle metriche tramite una web application. Permette di creare dashboard dinamiche interrogando le api di graphite-web. Nel mio caso ho organizzato una dashboard in modo da visualizzare sia i dati provenienti dai router da analizzate, sia gli anomaly score calcolati dal software di anomaly-detection ~\ref{fig:graphana}.


%todo: come vengono mandati i dati di ndpi?
Riassumendo i dati provenienti dai plugin di collect, dal mio plugin e da NDPI vengono collezionati da collectd e inviati tramite il plugin write\_graphite al backend go-carbon, che si occupa di ricevere i dati e salvarli sul file system. Per la visualizzazione dei dati viene usato grafana, che permette di visualizzare i dati richiedendo i dati alle carbonapi e contemporaneamente il mio tool di anomaly detection richiede i dati per analizzarli, ogni volta che ottiene dei risultati li manda a go-carbon per salvarli nel database ~\ref{fig:mos}.
% todo: approfondire MOS
% todo: fare immagine con la gestione dei dati


\section{Selezione features}

La scelta delle feature da utilizzare dipende da quale obiettivo su vuole ottenere, il mio obiettivo primario in questa tesi è di rilevare gli attacchi DDoS in uscita verso la sede centrale, quindi basandomi sugli attacchi più famosi e frequenti ho delineato una lista di parametri da osservare. Questi parametri sono:
% todo: capire le unità di misura i dati sono ogni secondo o ogni 10
\begin{itemize}
    \item \emph{bytes trasmessi al secondo}:questa metrica è utile, abbinata ad altre, per la rilevazione di attacchi che mirano alla saturazione della banda.
    \item \emph{pacchetti trasmessi al secondo}: questa metrica ha uno scopo simile alla precedente oppure aiuta ad avere informazioni sugli attacchi di tipo flooding.
    \item \emph{numero di connessioni aperte}: il numero di connessioni aperte è un indicatore importante per identificare tutti gli attacchi che mirano a saturare le connessioni di un server. %todo: questa ha molti usi
    \item \emph{numero di pacchetti con flag syn}: questa metrica è molto utile per la rilevazione di syn flood o port scanning.
    \item \emph{tls throughput}: tutto il traffico inviato su un canale sicuro tls non può essere analizzato più nello specifico, quindi viene raggruppato in questa categoria.
    \item \emph{dns throughput}: conoscere il traffico relativo al traffico dns può essere utile come indice di un attacco contro il server DNS aziendale oppure di un attacco di DNS amplification.
    \item \emph{ssh throughput}: potrebbe segnalare anomalie sull'utilizzo improprio di macchine nella sede centrale tramite delle connessioni ssh.
    %todo: da rivedere come funzionano gli icmp durante un syn flood
    \item \emph{icmp throughput}: è un indicatore di attacchi, per esempio durante un syn flood usando ip spoofing, con ip della sottorete dell'attaccante, al ritorno dei syn ack si vede un aumento degli icmp che indicano che l'host non è esistente o che la porta destinazione a cui è destinato il pacchetto è chiusa. Inoltre gli icmp possono anche essere usati direttamente per degli attacchi.
    \item \emph{ora del giorno}: l'ora del giorno viene aggiunta per caratterizzare al meglio il traffico lungo la giornata.
    %todo: aggiungere http e ntp
\end{itemize}

Tutte le feature vengono poi \underline{derivate su un tempo di 10/30s, per avere dei "rate"} confrontabili tra lavoro.

La scelta della features è nata da un compromesso tra i dati necessari per rilevare al meglio le anomalie, la riduzione dei dati da salvare sul server e di conseguenza l'uso di banda usata per il trasferimento. Inoltre un problema che ho dovuto tenere in considerazione è l'utilizzo di un \underline{acceleratore hardware} nei router Tiesse, il quale permette un incremento della velocità di routing, ma non permette di analizzare nel kernel i pacchetti.
 %todo: approfondire meglio come funziona il fast path
 
\section{Il mio tool}

% Cosa fa?
% Perchè l'ho fatto?

% Per rilevare le anomalie, una volta decise le features da analizzare, abbiamo dovuto decidere quale sistema dei precedentemente elencati adottare. La scelta è ricaduta sull'utilizzo di una rete neurale di autoencoder, la quale permette di allenare facilmente il modello in modo semi-supervisionato, grazie all'utilizzo della sola classe normale di traffico.

\subsection{Struttura}

% Posso parlare di come è strutturato il programma, che librerie ho usato 
Il programma è scritto in Python 3, con l'utilizzo di TensorFlow e Keras librerie per il ``machine learning'', effettua le richieste http tramite la libreria requests e gestisce i dati grazie a NumPy e Pandas: librerie per il calcolo matriciale e la gestione di tabelle e serie.
Il tutto, per assicurare una maggiore portabilità, ed assicurare la possibilità di effettuare il deploy su qualsiasi macchina, viene eseguito all'interno di un container Docker, la cui immagine viene creata e avviata tramite ``docker-compose''.


I dati all'interno del repository sono organizzati nel seguente modo:

\dirtree{%
.1 /.
.2 data\DTcomment{Cartella dove verrà salvato il modello e le immagini generate}.
.2 src\DTcomment{Cartella con i sorgenti del programma}.
.3 requirements.txt\DTcomment{Librerie di python necessarie}.
.3 utility.py \DTcomment{File in cui sono presenti le funzioni comuni}.
.3 update\_db.py \DTcomment{Funzioni per caricare i risultati su graphite}.
.3 evaluate.py \DTcomment{File con le funzioni per valutare se sono presenti anomalie}.
.3 train.py \DTcomment{File con funzioni per eseguire l'allenamento della rete}.
.3 model.py \DTcomment{File con funzioni per generare il modello}.
.3 test.py \DTcomment{Script che esegue la creazione di un modello, il train e la successiva valutazione di anomalie, basandosi su un file di impostazioni}.
.3 main.py \DTcomment{Script che esegue la creazione e il train del modello se non presente e a intervalli regolari verifica le anomalie}.
.2 test\_configs.
.3 test\_docs\_v3.json\DTcomment{Un esempio di file di configurazione usato per effettuare i test della validità del modello}.
.2 Dockerfile.
.2 docker-compose.yml.
.2 README.md.
}
\subsection{Keras e TensorFlow}

\paragraph{Tensorflow 2} è una libreria open-source sviluppata da Google per il machine learning che fornisce moduli ottimizzati per la realizzazione di algoritmi di machine learning e la loro esecuzione in maniera efficiente su CPU, GPU e TPU. Inoltre permette di scalare agevolmente su un'architettura con molti device.
% \cite{keras_about}

\paragraph{Keras} è un insieme di API ad alto livello di TensorFlow 2 che forniscono astrazioni essenziali per incentrarsi sulla facilità d'uso, la modularità e l'estendibilità per la risoluzioni di problemi relativi al machine learning, con una particolare concentrazione sui moderni problemi di deep learning.


\subsection{Elaborazione dei dati in input}
Come ricevo i dati dal server? Come li elaboro? Che api uso?
Genero tante piccole sequenze di 8 elementi (spiegare il motivo della scelta della sequence length => trade-off tra tempo di rilevamento dell'anomalia e mitigazione dei picchi e dei falsi positivi) e perchè lo faccio, come normalizzo i dati e perchè. Come fornisco i dati in input alla rete.


Le metriche relative alle applicazioni vengono aggiornate ogni 30 secondi, a differenza dei 10 secondi con cui aggiorno i dati collezionati da collectd, per questo motivo ogni volta che devo richiedere dei dati al server devo effettuare due richieste, tramite la libreria requests, alle carbonapi.

% (Viene usato http con basic auth, questo mi provoca un colpo al cuore lato sicurezza, ma lo posso giustificare dicendo che è un ambiente di test, in produzione come minimo verrà usato https.)

Un esempio di dati restituiti dalla prima richiesta è mostrata nel codice \ref{code:carbonapijson}, in cui per ridurre la quantità di dati da mostare nell'esempio, ipotizzo che l'intervallo di tempo richiesto sia di venti secondi.

\begin{lstlisting}[language=json,caption={Esempio di risposta delle carbonapi}\label{code:carbonapijson}]
[
    {
        "target": "Tiesse-EnvironmentPark.conntrack.conntrack",
        "datapoints": [
            [
                135,
                1619681610
            ],
            [
                134,
                1619681620
            ]
        ],
        "tags": {
            "name": "Tiesse-EnvironmentPark.conntrack.conntrack"
        }
    },
    {
        "target": "Tiesse-EnvironmentPark.interface-vdsl0_835.if_packets.tx",
        "datapoints": [
            [
                21.199091,
                1619681610
            ],
            [
                26.401902,
                1619681620
            ]
        ],
        "tags": {
            "name": "Tiesse-EnvironmentPark.interface-vdsl0_835.if_packets.tx"
        }
    },
    {
        "target": "Tiesse-EnvironmentPark.interface-vdsl0_835.if_octets.tx",
        "datapoints": [
            [
                4202.222148,
                1619681610
            ],
            [
                5862.413248,
                1619681620
            ]
        ],
        "tags": {
            "name": "Tiesse-EnvironmentPark.interface-vdsl0_835.if_octets.tx"
        }
    }
]
\end{lstlisting}
I dati ricevuti per essere elaborati devono essere memorizzati in una tabella con il corretto formato, vedi l'esempio di tabella \ref{table:tabella_dati_1}, per questo motivo i dati del json appena ricevuto vengono elaborati, sostituendo per prima cosa il timestamp, il quale viene trasformandolo da un unix timestamp ad uno che indica i secondi passati dalla mezzanotte e successivamente tutti gli elementi verranno aggiunti ad un dizionario con il corretto formato, come mostrato nell'estratto di codice \ref{code:get_data}.

% todo: e i syn?
\begin{table}[]

\begin{tabular}{||c c c c c||} 
\hline
index & timestamp  & .conntrack.conntrack & ..if\_packets.tx & ..if\_octets.tx \\ [0.5ex] 
\hline\hline
1 & 45810 & 71.0 & 7.500424 & 2416.445886 \\ 
\hline
2 & 45820 & 74.0 & 4.89923 & 51069.831629415 \\
\hline
3 & 45830 & 72.0 & 5.000281 & 1224.055123 \\
\hline
\end{tabular}
\caption{Tabella di esempio dei dati ricevuti relativi alle features di collect.}
\label{table:tabella_dati_1}
\end{table}
% \begin{lstlisting}[language=csv]
% ,timestamp,.conntrack.conntrack,..if_packets.tx,..if_octets.tx
% 0,45810,181.0,38.902328,6055.20679
% 1,45810,181.0,38.902328,6055.20679
% \end{lstlisting}

% todo: come metto le lettere accentate?
\begin{lstlisting}[language=Python, label={code:get_data}, caption={Funzione usata per scaricare i dati dal server}]
def get_data(s_time, e_time, hostname: str, interface: str,
    username: str = None, password: str = None):

    r = requests.get(
    f"http://{host}:{port}/api/datasources/proxy/4/render?"
    f"target={'&target='.join(target_list(hostname,interface))}"
    f"&from={s_time}&until={e_time}&format=json",
    auth=HTTPBasicAuth(username, password)
    )

    # se il valore ritornato non e' un'eccezione,
    # che verra' gestita a livelli superiori
    print(r.status_code, '-', r.url)
    if r.status_code != 200:
    raise Exception('Get features data error')


    # converto la risposta in un json 
    # e poi in un dizionario con chiave il target
    # e come valore i datapoints
    json_resp = r.json()

    # todo: probabilmente ci sono modi piu' efficienti
    data = data_to_dict(json_resp)
    data_list = list()
    for i in range(len(json_resp[0]['datapoints'])):
        line: dict = dict()
        timestamp =
        datetime.fromtimestamp(json_resp[0]['datapoints'][i][1])
        # trasformo il tempo in secondi dalla mezzanotte
        line['timestamp'] = timestamp.hour * 3600 \
                            + timestamp.minute * 60 \
                            + timestamp.second
        tl = target_list(hostname, interface)
        for e in tl:
            line[e] = data[e][i][0]
        data_list.append(line)

    # creo la matrice con pandas ed elimino le righe con valori
    # mancanti in rari casi le carbon api mi restituiscono dei
    # valori nulli, per questo motivo cancello le righe in cui
    # c'e' almeno un valore nullo
    matrix = pd.DataFrame(data_list).dropna()

    # successivamente richiedo anche i dati delle app, e quando
    # li ricevo, dopo averli manipolati, li unisco tutti
    # in un'unica tabella
    app_data_matrix =
    get_app_data(s_time, e_time, hostname, username, password)

    matrix = pd.merge(app_data_matrix, matrix)

    # questa funzione mi permette di ritornare l'elenco 
    # delle feature attualmente utilizzate, senza impostare
    # host e interfaccia da utilizzare
    matrix.columns = total_target_list('', '')
    return matrix

\end{lstlisting}

Ricevuti i dati e organizzati nel formato corretto, effettuo una nuova richiesta alle carbon api, per ricevere le statistiche relative alle applicazioni, il procedimento è simile a quello precedentemente descritto, l'unica differenza è che i dati sul database vengono salvati da ndpi ogni 30 secondi, per questo motivo devo estenderli mostrando lo stesso dato più volte, in modo da avere lo stesso numero degli elementi ottenuti precedentemente.

\begin{lstlisting}[language=Python]
for element in app_list:
    # se l'app non esiste nella risposta metto uno 0
    line[element] =
     data[element][i][0] if element in data else 0.0
# estendo i dati, replicando lo stesso dato ogni 10s per un totale di 30s
for j in range(3):
    line['timestamp'] += 10
    data_list.append(line.copy())
\end{lstlisting}

%todo: tabella con i dati di tutte le features

\paragraph{La standardizzazione dei dati} è la fase subito successiva, 
% differenziare i casi tra l'evaluate e il train
Durante il train calcolo la media e la deviazione standard sui dati per ogn, successivamente standardizzo e salvo la media e deviazione standard di ogni feature su un file.
Durante la fase di evaluate per standardizzare i dati uso i dati salvati precedentemente sul file e standardizzo i dati.
%todo: Un problema della standardizzazione dei dati è la gestione dei picchi le reti neurali lavorano meglio con dati piccoli, quindi i picchi potrebbero essere un problema, si può pensare un modo diverso per standardizzare i dati

\paragraph{Scelta della lunghezza delle sequenze dei dati.} Utilizzando i dati standardizzati genero ``N-K'' sequenze di lunghezza ``K'' elementi, dove ``N'' è il numero di elementi ricevuti precedentemente. Queste sequenze mi servono per uniformare i dati di allenamento e valutazione usando una lunghezza fissa per la sequenza di dati da dare in input alla rete neurale.
La scelta della lunghezza della sequenza è molto importante, perchè da essa dipende la capacità di rilevamento delle anomalie: più il valore sarà grande e più sarà preciso il riconoscimento delle anomalie e meno il programma sarà soggetto a falsi positivi, ma al tempo stesso renderà di difficile identificazione i picchi anomali e il tempo necessario per rilevare un'anomalia sarà maggiore.
%todo: aggiungo il codice per creare le sequenze?
%todo: qua spiego cosa ho scelto allegando delle immagini
Abbiamo scelto una lunghezza della sequenza di otto elementi, perchè analizzando il report \cite{imperva_ddos_report} il 25\% degli attacchi ha una durata inferiore ai 10 minuti e ...

% todo: citare https://keras.io/examples/timeseries/timeseries_anomaly_detection/
% todo: citare la lunghezza media degli attacchi \cite{ddos_kaspersky_q3_2020, imperva_ddos_report}
% todo: la sequenza si potrebbe allungare vedendo questi dati
\subsection{Modello della rete}

%Spiego come ho usato Keras per generarlo.
%todo: Spiego come sono arrivato a scegliere il modello e comparo i risultati tra 2/3 modelli totalmente diversi come numero di nodi, sarebbe interessante provare a mettere anche un modello con retroazione visto che stiamo usando sequenze temporali.

Il modello di una rete neurale serve a descrivere le interconnessioni, le diverse tipologie e la composizione dei livelli di una rete neurale, è un oggetto della libreria Keras, che può essere facilmente creato grazie alla funzione ``tf.keras.Model()''.
Nel nostro caso ogni livello prevedeva un solo livello in ingresso e uno in uscita, per questo motivo abbiamo usato il ``sequential model'' di Keras.
La nostra rete è composta da un semplice autoencoder a cinque livelli: uno di input e uno di output con dimensioni (lunghezza\_sequenza, numero\_features) e livelli con prima una riduzione e poi un incremento dei nodi, come nei classici autoencoders.
Avendo in ingresso dei dati con più dimensioni ho dovuto linearizzare i dati in ingresso (flatten) e ricostruire il vettore multidimensionale in uscita (reshape), nei nodi intermedi invece ho usato progressivamente un layer con 25 nodi, uno spazio latente con 8 nodi e un livello successivo di nuovo con 25. La scelta delle dimensioni dei livelli centrali è stata presa analizzando più configurazioni e considerando che usando una lunghezza della sequenza di otto elementi, con dieci features significa avere 80 nodi in input e output, di conseguenza i nodi nei livelli centrali devono essere meno.
Per creare la rete sono stati usati ``Dense layers'', in cui cui i nodi di due livelli sono interamente connessi tra loro e sono state usate come funzioni di attivazioni ``relu'' (Equazione \ref{eq:relu}) per i nodi intermedi e ``linear'', una funzione lineare, per i nodi dell'ultimo livello, questo permette di rappresentare anche i numeri negativi.
%https://keras.io/api/models/model/
Inoltre aggiungo dei ``Dropout layers'', un livello che imposta casualmente l'input del livello a zero, con una frequenza determinata, durante la fase di train, questo permette di ridurre il problema dell'overfitting.

\begin{equation} \label{eq:relu}
    f(x) = x^+ = max(0, x)
\end{equation}

Il modello viene infine compilato e salvato su file, in modo da potere essere usato successivamente.

\begin{lstlisting}[language=Python, label={code:get_data}, caption={Funzione usata per generale il modello della rete neurale}]
    sequential_model = keras.Sequential(
        [
            layers.Flatten(input_shape=shape),
            layers.Dense(25, activation='relu'),
            layers.Dropout(rate=0.2),
            layers.Dense(8, activation='relu'),
            layers.Dropout(rate=0.2),
            layers.Dense(25, activation='relu'),
            layers.Dropout(rate=0.2),
            layers.Dense(shape[1]*shape[0], activation='linear'),
            layers.Reshape((shape[0], shape[1]))

        ]
    )
\end{lstlisting}

\subsection{Train}
Il train deve essere effettuato su degli intervalli di tempo in cui non si sono verificate anomalie, per questo motivo viene fatta l'assunzione che tutto il traffico generato non presenti anomalie a meno di piccoli intervalli in cui è stato volutamente generato traffico malevolo per scopo di test. Volendo se l'amministratore di sistema notasse altri periodi di traffico anomalo potrà escluderli dai dati di allenamento.
Inoltre per avere un maggiore numero di dati, e velocizzare l'apprendimento dell'allenamento della rete, è possibile utilizzare i dati provenienti da più router delle sedi periferiche, se ipotizziamo che il traffico generato sia simile tra loro.
Dalla selezione degli intervalli di tempo su cui si vuole effettuare il train per semplicità in questa prima versione di algoritmo vengono esclusi i weekend: giorni della settimana in cui, nel nostro caso, il traffico è molto diverso dagli altri, volendo risolvere il problema si potrebbe usare una seconda rete da usare solo nei giorni festivi o %todo: proporre altre soluzioni per i weekend
Nella fase di train per prima cosa verrà effettuato la raccolta e la trasformazione dei dati dal server, come spiegato precedentemente e avere caricato da file il modello generato.
Successivamente prima di effettuare il train vero e proprio, grazie alla funzione ``train\_test\_split'' della libreria sklearn i dati vengono divisi in due insiemi, quello di train e quello di test, con il 15\% dei dati usati per il test e il restante per il train.
L'allenamento viene effettuato usando il metodo ``fit'' della classe model e usando in input sia per le x, che per le y, l'insieme di train appena generato e l'insieme di test per validare il modello. Inoltre è abilitato l'EarlyStopping, in cui se per più di cinque iterazioni non si hanno miglioramenti, viene stoppato il processo di allenamento, questo serve per ridurre i tempi di train e ridurre la possibilità di overfitting.
%todo: da valutare possibilità di overfitting.

Come spiegato precedentemente, l'utilizzo degli autoencoders permette di ottenere degli ``anomaly score'' per ogni sequenza di dati, quindi terminata la fase di allenamento della rete devo decidere sopra quale valore devo considerare i dati anomali.
Le soglie sopra i quali considero i dati anomali vengono prese dal massimi valore degli anomaly score per ogni feature: avendo ipotizzato che tutti i dati forniti per il train non siano anomali, devo assicurarmi che effettuando la valutazione delle anomalie su quei dati nulla risulti anomalo.
% todo: potrei anche aumentare le soglie di un certo valore per essere meno soggetto a falsi positivi
Per calcolare l'anomaly score calcolo il valore medio dei valori assoluti delle differenze tra il valore originale e quello ricostruita.
\begin{equation}
    % \caption{Formula per il calcolo dell'anomaly score}
    anomaly\_score = \frac{\sum_{elementi\_sequenza}\lvert valore\_originale - valore\_ricostruito \rvert}{lunghezza\_sequenza}
\end{equation}
Le soglie calcolate vengono salvate su un file, da potere usare successivamente per la valutazione delle anomalie.

Quando eseguire il train?
% todo: per quanto tempo faccio il train? 

%todo: Immagini di confronto dei valori ricostruiti e originali con anomalie e non


% Come effettuo il train?
% Quali dati uso?
% Cosa escludo? 
% Per quanto tempo?
% e se il traffico varia nel tempo?
% Vantaggi e svantaggi del train fatto in questo modo
% Overfitting? 

% Assumo che gli intervalli di traffico utilizzati per il train non includa anomalie, que
% Utilizzo il traffico di più uffici per velocizzare l'apprendimento se hanno caratteristiche simili

% Scartati i weekend perchè i dati sono molto diversi rispetto al resto della settimana e la rete non tiene in considerazione il giorno della settimana.

\subsection{Evaluate}
% Come effettuo la valutazione? Su quali sequenze?
Allenato il modello a ricostruire l'input e calcolate sopra le quali considerare gli anomaly score anomali è possibile verificare se il traffico generato in un intervallo di tempo da un determinato router è anomalo.  
Per mettere in atto questa fase, dopo avere trattato i dati in ingresso nello stesso modo dei dati per l'allenamento, leggo dal file generato precedentemente le soglie e verifico se ogni sequenza di k elementi è anomala, calcolando gli anomaly score come effettuato nell'ultima fase del train. Se gli anomaly score superano le soglie si è verificata un'anomalia, a questo punto viene segnalata all'amministratore di rete e viene attivata la fase di mitigazione sul router.
% come viene attivata questa fase? C'è una comunicazione tra server e router?
Questa fase viene ripetuta automaticamente dal software ogni periodo di tempo definito (nell'ordine dei trenta secondi) per i dati di ogni router di cui si vogliono monitorare le anomalie.

\section{Test sulle anomalie}

L'analisi delle anomalie in questa tesi ha l'obiettivo principale di rilevare attacchi DDoS, quindi basandoci sullo studio degli attacchi più popolari abbiamo selezionato un ristretto elenco di attacchi possibili da riprodurre:
%todo: noi l'abbiamo fatto da una sorgente sola, quindi alla fine è un dos, dobbiamo aggregare maggiormente i dati per capire se si tratta di un ddos?
%todo: la feature ssh non sono convinto che sia utile, nel nostro caso è usata talmente poco che temo sia quasi dannosa (viene usata per ricostruire meglio le altre metriches)
\begin{itemize}
    \item SYN flood
    \item ICMP flood
    \item UDP flood
    \item DNS flood
    \item DNS amplification
\end{itemize}

\subsection{Tool utilizzati}

Per effettuare le varie tipologie di attacchi sono stati usati software open-source reperibili sul web e sono stati scritti dei tool per adeguarsi al meglio alle nostre esigente.

\paragraph{hping3} è un tool in grado di generare pacchetti di rete TCP/IP personalizzati. Il nostro utilizzo è stato per generare attacchi di tipo SYN flood, ICMP flood e UDP flood. Il tool, oltre a permettere di generare i pacchetti personalizzati da mandare alla vittima, permette di regolare la portata dell'attacco, tramite le possibilità di sceltà del rate a cui inviare i pacchetti.
% https://linux.die.net/man/8/hping3


Per cosa è stato usato, quali sono i limiti di hping3 e perchè dobbiamo

\paragraph{Il nostro toool}

Hping3 non consente di riprodurre tutte le tipologie di attacco a noi necessarie, per questo motivo è stato sviluppata un'applicazione per costruire pacchetti di tipo DNS da utilizzare per creare attacchi di tipo DNS flood e DNS amplification.
Il tool è sviluppato in C++, per una maggiore efficienza e per raggiungere un maggiore throughput durante gli attacchi, utilizzando la libreria ``libtins''\footnote{ http://libtins.github.io/}.
Il programma genera delle DNS query indirizzate verso i DNS server configurati in un file di configurazione. A scopo di test, per non sovraccaricare il server DNS aziendale, abbiamo utilizzato dei Raspberry Pi, con in esecuzione dei container di CoreDNS: semplici istanze DNS configurate in modo da mantenere in cache le risposte alle richieste effettuate e abbiamo effettuato solo richieste di risoluzione di un particolare dominio da parte del software di attacco.
Inoltre per semplicità d'uso il software permette anche di effettuare attacchi di tipo syn flood con ip spoofing.

% todo: immagini di wireshark / gestione attività durante gli attacchi

\subsection{Risultati}

\chapter{Mitigazione degli attacchi}


\section{Introduzione}
Prova 
\subsection{Bloccare l'ip spoofing}
L'ip spoofing permettere di usare la tecnica dell'amplification


\section{Funzionamento}
Prova prova
\subsection{eBPF}
\subsection{BCC}

Qua posso parlare di due alternative, la prima è riutilizzare il sistema di anomaly detection simile a quello presentato precedentemente elencando tutti i problemi e i vantaggi.

Il secondo consiste nel raccogliere i dati come prima, e creare un ranking per ogni feature risultata anomala precedentemente e a quel punto blocco i flussi sopra una certa soglia, ma quale?

Mentre per l'ip spoofing come la gestisco?

\section{Test sulle anomalie}
\subsection{Tool utilizzati}
\subsection{Risultati}

\chapter{Lavoro futuro}

% Ciao \cite{dirac}
\chapter{Conclusioni}

Sicuramente una cosa da migliorare è la sicurezza per assicurarsi che neanche le sedi periferiche non siano infettate


\listoffigures
% \begin{thebibliography}{9}
    \bibitem{ddos_survey_1} Saman Taghavi Zargar, James Joshi and David Tipper {\em A Survey of Defense Mechanisms Against Distributed Denial of Service (DDoS) Flooding Attacks}, 2013. ()
s
    \bibitem{ddos_survey_2} s

    \bibitem{ddos_motivations} Tasnuva Mahjabin, Yang Xiao, Guang Sun and Wangdong Jiang {\em A survey of distributed denial-of-service attack, prevention, and mitigation techniques}, 2017. (https://journals.sagepub.com/doi/10.1177/1550147717741463)

    \bibitem{ddos_survey_3} Neha Agrawal and Shashikala Tapaswi {\em Defense schemes for variants of distributed denial-of-service (DDoS) attacks in cloud
    computing: A survey}, 2017. (https://doi.org/10.1080/19393555.2017.1282995)

    \bibitem{ddos_survey_4} Jasmeen Kaur Chahal, Abhinav Bhandari and Sunny Behal {\em Distributed Denial of Service Attacks: A Threat or Challenge}, 2019. (https://doi.org/10.1080/13614576.2019.1611468)

    \bibitem{ddos_kaspersky} kaspersky securitylist.com {\em DDoS Report - DDoS attacks in Q4 2020
    }, 2020. (https://securelist.com/ddos-attacks-in-q4-2020/100650/)
    
    \bibitem{ddos_kaspersky_q3_2020} kaspersky securitylist.com {\em DDoS Report - DDoS attacks in Q3 2020
    }, 2020. (https://securelist.com/ddos-attacks-in-q4-2020/100650/)

    %% da qui va cancellata
    % \bibitem{gal} G.~Galilei, {\em Nuovi studii sugli astri medicei}, Manuzio, Venetia, 1612.

    % \bibitem{tor1} E.~Torricelli, in ``La pressione barometrica'', {\em Strumenti

    %         Moderni}, Il Porcellino, Firenze, 1606.
    % \bibitem{tor2} E.~Torricelli e A.~Vasari, in ``Delle misure'', {\em Atti Nuovo
    %         Cimento}, vol.~III, n.~2 (feb. 1607), p.~27--31.

    % \bibitem{duane1964} Duane J.T., \emph{Learning Curve Approach To Reliability Monitoring}, IEEE Transactions on Aerospace, Vol. 2, pp. 563-566, 1994

    % % altri riferimenti da usare come esempi.

    % \bibitem{chiesa2008} Chiesa S., \emph{Affidabilità, sicurezza e manutenzione
    %     nel progetto dei sistemi}, CLUT, gennaio 2008
    % \bibitem{chiesa2}Chiesa S., Fioriti M., Fusaro R., \emph{On Board System
    %     Technological  Level Improvement Effect on UAV MALE}
    % \bibitem{bigliano2010} Bigliano M., \emph{Sicurezza nell'installazione di un velivolo
    %     senza pilota MALE; applicazione di metodologia di Zonal Safety
    %     Analysis al velivolo del Progetto SAvE}, Politecnico di Torino,
    % maggio 2010
    % \bibitem{astrid2012} Chiesa S., Di Meo G.A., Fioriti M., Medici G., Viola N.,
    % \emph{ASTRID - Aircraft on board Systems sizing and TRade-off
    %     analysis in Initial Design}, Research Bulletin, Warsaw University
    % of Technology, Institute of Aeronautics and Applied Mechanics,
    % p. 1-28, 17-19, ottobre 2012

\end{thebibliography}

\printbibliography
\end{document}
