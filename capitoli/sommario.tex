% Si veda la documentazione per verificare la differenza fra abstract
% e summary. Perciò se se ne usa uno, non si deve usare l'altro.
% \english
% \begin{abstract}
% This short abstract, is typeset with the \texttt{abstract} environment (from the \texttt{report} document class) just to test if it works. or what concerns working, it works, but in Italian ist title turns out to be ``Sommario'' in bold face series and normal size; its apperance looks like  a bad copy of what one obtains with the \texttt{\string\summary} command. In English, though, its title is “Abstract”, as it should be, since at the beginning of this template a suitable \texttt{\string\ExtendCapions} command was issued.

% Please, read the documentation  in Italian (file \texttt{toptesi-it.pdf} in order to fully understand the difference beteesn “abstract” and “summary” in the context of this bundle.

% \end{abstract}

% Fine dell'altro esperimento
\italiano
\sommario

Gli attacchi di denial of Service distribuiti (DDoS) sono uno dei maggiori problemi di sicurezza delle reti. Hanno lo scopo di impedire ad utenti legittimi l'accesso a dei servizi o degradare loro le prestazioni. \uline{Contestualmente, gli strumenti a disposizione di chi deve mitigare questo tipo di attacchi evolvono con l'affinamento delle tecniche di riconoscimento del traffico e con la capacità di monitorare le reti aziendali esposte su Internet.} La questione di ottenere sempre maggiore padronanza della propria infrastruttura IT si scontra continuamente con l'aumentare del numero e con la continua evoluzione degli applicativi, della tecnologia e dell'espansione complessiva della rete Internet stessa. In questa tesi proveremo a identificare anomalie riconducibili ad attacchi DDoS, in un contesto di una rete aziendale con più sedi, usando un riconoscimento delle anomalie effettuato tramite  una rete neurale allenata su dati provenienti dai router di più sedi aziendali e una successiva mitigazione degli attacchi tramite un agent sugli stessi. Approfondiremo l'uso di una particolare tipologia di rete neurale chiamata autoencorder, che si sta diffondendo rapidamente sia in ambito accademico che industriale nell'ambito dell'anomaly-detection. Esporremmo il nostro contributo prendendo come caso d'uso dati raccolti da una sede distaccata della società Tiesse s.p.a., produttrice di router e altri dispositivi di rete, coinvolta nello studio e nella realizzazione di questo progetto.
