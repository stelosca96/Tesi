\chapter{Lavoro futuro}

La soluzione qui sviluppata è stata utilizzata in un ambiente di test, per portare questa applicazione ad uno stato di produzione è necessario introdurre un meccanismo per inviare i segnali tra i molteplici router che hanno comportamento di client e il server più prestazionale che si occupa di calcolare gli anomaly score al verificarsi delle anomalie. Per sviluppare questa comunicazione è possibile utilizzare una connessione TLS diretta tra il client e il server.

Altre funzioni da implementare per ottenere un sistema totalmente funzionante in produzione è un'interfaccia grafica e un sistema di notifiche, tramite i quali l'amministratore di sistema potrà gestire il sistema ed essere allertato in caso di problemi. 

% ma per maggiore comodità in caso in caso di creazione di un'ulteriore interfaccia grafica di amministrazione, potrebbe essere


Per quanto riguarda un'applicazione alternativa di questo sistema di anomaly detection potrebbe essere interessante allenare il modello a riconoscere uno specifico pattern di un'applicazione, in modo da potere identificare tale flusso tra tutto il traffico in transito dal router, trasformando l'anomaly score in un "similarity" score.

Un'altra possibile applicazione futura potrebbe essere aggregare il traffico di tutte le sedi, per avere una panoramica ancora maggiore in caso di attacchi a basso low rate (vedi sezione \ref{paragraph:ddos_degradation}) difficilmente riconoscibili singolarmente.